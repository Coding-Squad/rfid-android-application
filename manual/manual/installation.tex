\chapter{SYSTEM INSTALLATION AND CONFIGURATION}

\section{User Requirements}
\begin{itemize}
	\item Android device (phone or tablet) capable of:
	\begin{itemize}
		\item Taking photos
		\item Using Bluetooth
		\item Connecting to the internet
	\end{itemize}
	\item TSL 1128 Bluetooth\textsuperscript{\textregistered} RFID Scanner
	\item RFID tags for all items that will be inventoried
	\item Internet connection
	\item Google Account
	\item GitHub Account
\end{itemize}

\section{Installation}
\begin{enumerate}
	\item For devices running a version of Android earlier than 8.0 (Oreo),
please go to \textbf{Settings} $\rightarrow$ \textbf{Security} and make sure that \textbf{Unknown Sources} is \textbf{enabled}.
Devices running Android 8.0 or later have a modified security model that will prompt on a per-application basis.

	\item Using a web browser on the device, navigate to \url{https://github.com/CS4560-4561/rfid/releases}.
	
	\item Download and install the latest \textbf{.apk} file from the website.
\end{enumerate}

\clearpage
\section{Configuration}
After the program is installed on the device, pair the TSL 1128 Bluetooth scanner with the device.
The procedure varies based on device; consult your manual for specific procedures. \\

\noindent
Once the scanner is paired, open the application and log in with your Google account. \\

\noindent
To configure the scanner, go to the \textbf{menu} $\rightarrow$ \textbf{Settings} $\rightarrow$ \textbf{Scan and Connect TSL Device} and select the device as shown in the list of paired devices. This configures the device for use.

\noindent
Currently (v0.1.0) there is a limitation requiring re-connecting the device every time the application is restarted.

\vspace*{\fill}
{\raggedleft{\scshape James Underwood} \\ Athens, OH \\ February 2018\par}
